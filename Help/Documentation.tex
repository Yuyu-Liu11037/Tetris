\documentclass{ctexart}
\usepackage[utf8]{inputenc}
\usepackage{hyperref,geometry}
\geometry{margin=1in}
\hypersetup{
    colorlinks=true,
    linkcolor=black,
    filecolor=magenta,      
    urlcolor=black,
}

\title{Documentation}
\author{LIU Yuyu}
\date{May 2022}

\begin{document}

\maketitle
\tableofcontents

\newpage
\section{代码讲解}
\subsection{Block.java}
Serializable接口允许程序员把内存里已经实例化了的对象当成文件保存下来,比如游戏存档。


\subsection{BlockFactory.java(存放方块)}
列举出所有可能的方块形状:每种方块的初始形状以及3种旋转过后的形状。每种方块都由4个小正方形组成,并且都有4种旋转后的形状(目的是为了代码的统一)。


\subsection{StartGUI.java(开始游戏这个界面)}
\begin{enumerate}
    \item java.awt.Window.setVisible(boolean): 显示/隐藏GUI组件。setVisible(true)方法的意思是说数据模型已经构造好了,允许JVM可以根据数据模型执行paint方法开始画图并显示到屏幕上了。(并不是显示图形,而是可以运行开始画图。要把setVisible()方法放到最后面,因为代码是按顺序执行的,如果放前面,之后添加其它组件的时候,有可能会显示不出来。)
    \item JComboBox:下拉列表框。以下拉列表的形式显示多个选项,用户可以从下拉列表中选择一个值。
    \begin{itemize}
        \item DefaultComboBoxModel<>: 此类继承了AbstractListModel抽象类,实现了ComboBoxModel接口,可以很方便地做到动态更改JComboBox的项目值。
        \item BorderLayout布局管理器将界面分为5个部分。
    \end{itemize}
    \item 构造方法StartGUI():
    \begin{itemize}
        \item setDefaultCloseOperation(JFrame.EXIT\_ON\_CLOSE): "EXIT\_ON\_CLOSE": 点击窗口左上角时,结束应用程序,退出JVM。
        \item setBounds(x,y,width,height):背景面板区域。
        \item setBackground:背景颜色
        \item setBorder:边框
        \item setLayout: 对当前组件设置为流式布局(使用百分比设置宽高的布局,随着设备屏幕的改变,容器的宽高、位置变化)。
        \item setContentPane(contentPane): 在JFrame对象中可以添加swing组件。JFrame不是一个容器,只是一个框架,不能直接用add方法添加组件。对JFrame添加组件的一种方法是 建立一个JPanel中间容器,把组件添加到中间容器中,用setContentPane()方法把该容器设置为JFrame的内容面板。
    \end{itemize}
\end{enumerate}


\subsection{TetrisFrame.java}
\begin{enumerate}
    \item TetrisFrame(int level): 选择不同难度.
    \item initScores(): 读取存档。ObjectInPutStream可用于将InputStream转化为Object, 这个过程称为反序列化。
    \item showPause(): 暂停界面
    \item showGameOver(): 游戏结束界面
    \begin{itemize}
        \item JScrollPane是一个滚动面板,只能添加一个组件,但可以添加一个面板JPanel再把很多组件添加到JPanel上。
        \item JTable是用来显示和编辑常规的二维单元表。
    \end{itemize}
    \item saveScore(): 保存本次分数,结束时显示
    \item saveProgress(): 存档
\end{enumerate}

\subsection{TetrisModel.java}
\begin{enumerate}
    \item 构造方法TetrisModel(): 
    \item initialize(): 随机生成一个新的俄罗斯方块。
    \item rotete(),toLeft(),toRight().
    \item down(): 判断是否能继续下落,若能,下落;若不能,停止下落并去掉方块内的分割线(与可移动的方块区分)。
    \item check(): 判断是否能消除一整行并且加分。
    \item resetColor(): 重新生成俄罗斯方块。
\end{enumerate}


\end{document}
